% Introduction
\section{Introduction}
%Intro on the research topic of finding a generalist morphology and controller pair for diverse environment (wie wat waar wanneer waarom en hoe).
Since Karl Sims' publication in 1994, 'Evolving Virtual Creatures' \cite{Karl_Sims_1994}, demonstrated virtual creatures interacting within a simulated three-dimensional physical environment, numerous researchers have pursued simultaneous co-evolution of both morphology and control \cite{Nick_Cheney_2017,Emma_Stensby_2021,Joshua_Auerbach_2014,Luis_2024}. By co-evolving both morphology and control, researchers can develop more effective AI agent systems for various tasks. This approach leverages the principle of embodied cognition, which posits that intelligence arises not solely from the brain or an agent's control system, but from the dynamic interaction between the brain, body, and environment \cite{Josh_Bongard_2013}.

Unfortunately, an evolved morphology-controller pair will still specialize on the environment it has been trained in. This specialization occurs because the evolutionary process fine-tunes both the morphology and controller to the specific conditions of the training environment, resulting in high task performance in only the training environments but reduced task performance to new or marginally different environments. In the real-world, environments are inherently dynamic and unpredictable. Factors such as changing weather conditions, varying terrains, and unforeseen obstacles can significantly alter the operational context of an agent. This variability poses a substantial challenge to the robustness and generalizability of evolved morphology-controller pairs. For instance, an agent trained to navigate a smooth indoor surface may struggle when faced with outdoor terrains that include gravel, mud, or steep inclines. Similarly, agents designed for static environments might fail to adapt to environments with moving obstacles or varying lighting conditions.

To overcome this limitation, it is crucial to focus on developing robustness and generalizability in evolved agents. Robustness refers to the ability of an agent to maintain desirable behavior despite variations or perturbations in its input and output data \cite{Ravi_Mangal_2019, Charles_Packer_2019, Xu_Mengdi_2022}. Generalizability, on the other hand, refers to the ability of an agent to maintain desirable behavior under different conditions to those encountered during training \cite{Charles_Packer_2019,Xu_Mengdi_2022}. Co-evolved morphology-controller pairs have been shown to generalise well to environments similar to those encountered during training. However, comprehensive studies examining the robustness and generalizability of co-evolved morphology-controller pairs remain sparse in the existing literature.

\textbf{In this paper, we perform an extensive analysis on ...BLABLABLA}


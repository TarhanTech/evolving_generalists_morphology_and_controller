\section{Introduction}
The co-evolution of both morphology and control was first demonstrated by Karl Sims in his 1994 publication, "Evolving Virtual Creatures" \cite{Karl_Sims_1994}, where he demonstrated virtual creatures interacting within a simulated three-dimensional physical environment. This initiated the pursuit by numerous researchers of the simultaneous co-evolution of both morphology and control \cite{Cheney_2017,Emma_Stensby_2021,Joshua_Auerbach_2014,Luis_2024}. By co-evolving both morphology and control, researchers can develop more effective AI agent systems for various tasks. This approach leverages the principle of embodied cognition, which suggests that intelligence arises not solely from the brain or an agent's control system, but from the dynamic interaction between the brain, body, and environment \cite{Josh_Bongard_2013}.

Moreover, co-evolving a morphology-controller pair (MC-pair) provides major benefits in robotic development, such as the ability to avoid the conventional physical design stage. Co-evolution enables the search for a potential morphological structures by evolving the MC-pair in a simulated environment, thereby avoiding the time of iteratively redesigning and testing the robot in the physical world. Using an approach like this can furthermore speed up the development process and encourage the creation of more innovative morphologies that might not have been thought of in more traditional designs or initially deemed plausible.

Unfortunately, an evolved MC-pair will still specialize on the environment it has been trained in. This specialization occurs because the evolutionary process fine-tunes both the morphology and controller to the specific conditions of the training environment, resulting in high task performance in only the training environments but reduced task performance to new or marginally different environments. In the real-world, environments are inherently dynamic and unpredictable. Factors such as changing weather conditions, varying terrains, and unforeseen obstacles can significantly alter the context of an agent. This variability poses a challenge to the robustness and generalizability of evolved MC-pairs. For example, an agent accustomed to smooth interior surfaces would find it difficult to maneuver over muddy, gravelly, or steeply inclined outdoor terrains. Agents created for static surroundings may also not be able to adjust to environments with changing lighting or moving impediments. Therefore, research into evolving a generalist MC-pair that can operate effectively across a wide range of environmental settings is critically important.

To overcome this limitation, it is crucial to focus on developing robustness and generalizability in evolved agents. Robustness refers to the ability of an agent to maintain desirable behavior despite variations or perturbations in its input and output data \cite{Ravi_Mangal_2019, Charles_Packer_2019, Xu_Mengdi_2022}. Generalizability, on the other hand, refers to the ability of an agent to maintain desirable behavior under different conditions to those encountered during training \cite{Charles_Packer_2019,Xu_Mengdi_2022}. 

In this paper, we investigate the co-evolution of a generalist MC-pair for a wide range of environments, utilizing the Farama Gymnasium's Ant-v4 task \cite{Gymnasium2023} as our testing framework. We modified it to allow for the evolution of both the leg lengths and widths of the ant, thereby adapting its morphology to suit various environments. Our training methodology employs an incremental variable training schedule, which is heavily based on the training algorithm specified by Triebold et al. \cite{Corinna_Triebold}, with minor modifications tailored to our objective of evolving a generalist MC-pair for a wide range of environments. Our results show that the method employed in this paper improves generalizability, enabling the MC-pair to perform well on environments not seen during training. Furthermore, while specialist MC-pairs can excel in performance, this is not always the case.
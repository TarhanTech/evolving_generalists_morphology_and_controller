\section{Conclusion}
    This work has presented a comprehensive investigation of the co-evolution of morphology and control, highlighting its benefits, for instance, eliminating the need to design the morphology prior to training and the potential to evolve more innovative and performant agents. However, it also addresses the challenges associated with evolving MC-pairs, such as premature convergence and embodied cognition. We examined the possibility of evolving an MC-pair that can generalize across a wide range of environments, thereby eliminating the need for a specialist MC-pair for every specific use case. To achieve this, we employed the XNES evolutionary strategy, optimizing both morphology and ANN parameters simultaneously, in conjunction with an incremental and variable training schedule.

    The results from the experiments suggest that the method employed in this study improves generalizability, enabling the MC-pair to perform well in both interpolation and extrapolation testing. Furthermore, while specialist MC-pairs can excel in certain environments, this is not always the case. In many environments, especially challenging ones, they perform less effectively than the generalist. These findings underscore the importance of incorporating variability into the learning process, particularly when evolving MC-pairs.
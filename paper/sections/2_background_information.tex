% Background Information
\section{Background Information}
\subsection{Co-evolution of morphology and control}
Talk about current research of co-evolution of morphology and control
Embodied cognition and how to tackle this problem
What are issues and solutions on co-evolution.


\subsection{Robustness and generalizability}
Robustness refers to the ability of an agent to maintain desirable behavior despite variations or perturbations in its input and output data \cite{Ravi_Mangal_2019, Charles_Packer_2019, Xu_Mengdi_2022}. Consequently, a robust agent is less susceptible to input and output perturbations. This makes robustness a critical attribute, as inputs and outputs from the training environment can differ significantly from those in the testing environment, especially in real-world scenarios. For instance, an input perturbation can be caused by a sensor defect, resulting in slight measurement errors. In reinforcement learning, this means that we need to build robustness against the uncertainty of state observations and the actual state. Similarly, an output perturbation can result from a motory issue of the agent. In reinforcement learning, this means that we need to build robustness against uncertain actions between the actions generated by the agent and the conducted actions \cite{Xu_Mengdi_2022}. On the other hand, generalizability refers to the ability of an agent to maintain desirable behavior under different conditions to those encountered during training \cite{Charles_Packer_2019,Xu_Mengdi_2022}. It assumes correct input and output data and is concerned on the actual differences between the training environment and the testing or production environment.